\section{Ventajas y Desventajas}

Como se ha podido apreciar en el desarrollo de este trabajo, \emph{TF-IDF} es
un eficiente y simple algoritmo para hacer coincidir palabras en una consulta
con documentos que son relevantes para esa consulta. Como cada
modelo de RI, este tiene aspectos que lo hacen beneficioso, y al mismo tiempo,
a pesar de la fuerza del modelo, se escapan algunos puntos que le imponen
limitaciones.\cite{ramos, aparicio, mishra}

\subsection{Ventajas}

\subsubsection{Esquema de Ponderación}

El modelo vectorial es muy versátil y eficiente a la hora de generar
\emph{rankings} de precisión en colecciones de gran tamaño, lo que le hace
idóneo para determinar la equiparación parcial de los documentos. De esta
forma, aunque se recuperen \emph{n} documentos relevantes a una consulta, la
misma relevancia viene dada por una métrica de \emph{ranking} y tendremos, de
cierta manera, ''relevancia dentro de la relevancia''.

\subsubsection{Coincidencia Parcial}

La estrategia de coincidencia parcial permite la recuperación de documentos
que se aproximen a los requerimientos de la consulta. De esta forma quizas se
tengan documentos recuperados que no cumplan con todas las características de
la consulta, sino una parte considerable de la misma, los cuales son relevantes
a la consulta hecha. Esto se logra teniendo en cuenta los pesos tf-idf para
determinar la representatividad de los documentos de la colección.

\subsubsection{Medida de Similitud}

Se demuestra la eficacia de la \emph{fórmula del coseno} como métrica que
ordena los documentos de acuerdo al grado de similitud con la consulta,
aprovechádose de resultados \emph{geométrico-algebráicos} que plantean la
similitud entre vectores de acuerdo al coseno del ángulo que ellos determinan.
Se tiene en cuenta la intersección de los términos del documento con respecto
a la consulta, por lo que la gradación de los resultados es más precisa.

\subsection{Desventajas}

\subsubsection{Estructura del Lenguaje}

Al ser un modelo estadístico-matemático, no tiene en cuenta la estructura
sintáctico-semántica del lenguaje natural. En general recupera documentos
cuando hay igualdad de palabras entre el documento y la consulta no tiene en
cuenta, por ejemplo, sinónimos. Esto limita en gran medida la fortaleza del
modelo, teniendo situaciones donde, por la propia estructura del lenguaje, una
consulta con un valor semántico que otra, no tenga los resultados esperados,
aunque se refieran a temas relacionados.

\subsubsection{Umbral de Relevancia}

Establecer un umbral para definir que documentos son relevantes y cuales no es
un valor que no es fijo y fiable a cada colección, o sea, para cada colección
hay que probar y reajustar para ver que umbral se ajusta mejor y permite tener
métricas de evaluación más compensadas.

\subsubsection{Independencia de términos}

El modelo asume la independencia mutua entre los términos indexados, cuando,
como suele suceder en la actualidad, muchas veces existe una correlación
evidente entre los propios términos. Por ejemplo, es muy probable que en
textos científicos las palabras \emph{sujeto} y \emph{espécimen} aparezcan
altamente relacionadas.
