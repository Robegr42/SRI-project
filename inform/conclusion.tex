En el trabajo presentado se mostraron las características de un SRI utilizando
una implementación de un \emph{Modelo Vectorial (tf-idf)}. Se explica su
diseño, algoritmos utilizados, uso de varias colecciones de prueba para su
evaluación, así como el uso de aspectos interesantes como la retroalimentación
del sistema, el cual, sin duda alguna constituye un aspecto novedoso de dichos
sistemas. Se debatieron, además, los principales beneficios y limitantes que
existen en el uso de estos.

La implementación del SRI mostrado dio resultados positivos, la sencillez del
modelo hace que su eficacia pruebe la fortaleza de este algoritmo. Aunque se
evidencia que este modelo trabaja efectivamente para colecciones de tamaño
razonable, en el mundo cambiante y con cada vez más información para procesar
es necesario seguir mejorando el proceso de RI, si con la retroalimentación
existe una mejora en el rendimiento, de algo podemos estar seguros es que en
lo adelante necesitaremos algoritmos más poderosos y muchas otras mejoras en
estos mismos modelos para estar a la altura de la creciente era de la
información.