\section{Conclusiones}\label{sec:conc}

En el trabajo presentado se mostraron las características de un SRI utilizando
una implementación de un modelo vectorial (TF-IDF). Se explica su diseño,
siendo este, uno de las esquemas más usados en la actualidad. Se detallan las
distintas etapas del proceso desde la realización de una consulta, y cómo esta
es procesada, hasta la recuperación de los documentos relevantes. Se explican
los algoritmos utilizados en la creación del modelo así como el uso de varias
colecciones de prueba para su evaluación. Se debatieron, además, de forma
ctítica, los principales beneficios y limitantes que existen en el uso de estos
sistemas, los cuales permiten analizar posibles mejoras a realizar en futuras
investigaciones.

Se pudo apreciar en la evaluación de los resultados cómo para modelos con
diferentes resultados de efieciencia, los valores de relevancia son muy similares.
Esto demuestra que este valor en sí no es una buena métrica para la estimación
de la eficacia de un modelo.

También se analizó el impacto de utilizar diferentes configuaciones mediante el uso
de herramientas como lematización y stemming. Se pudo observar una pequeña mejora
en la eficiencia de los modelos obtenidos. Sin embargo, la misma fue casi
imperceptible. Esto da una idea de lo difícil que puede ser mejorar estos tipos
de modelos. Por ello, para alcanzar una mayor eficiencia, los modelos en la 
actualidad deben hacer uso de otras herramientas como retroalimentación.

Finalmente, se demuestra la facilidad de utilizar Python para la creación de
sistemas de recuperación de textos. Con la utilización de librerías como
\emph{nltk}, \emph{numpy} y \emph{matplotlib} se pueden realizar diversos
trabajos de análisis de texto y mostrar los resultados, de una manera eficiente
y sencilla.
