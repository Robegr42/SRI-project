\section{Conclusiones}\label{sec:conc}

En el trabajo presentado se mostraron las características de un SRI utilizando
una implementación de un \emph{Modelo Vectorial (tf-idf)}. Se explica su
diseño, siendo este, uno de las esquemas más usados en la actualidad. Se
detallan las distintas etapas del proceso desde la realización de una consulta,
y como esta es procesada, hasta la recuperación de los documentos relevantes a
la propia consulta. Se explican los algoritmos utilizados en la creación del
modelo así como el uso de varias colecciones de prueba para su evaluación.
Se debatieron, además, de forma ctítica, los principales beneficios y
limitantes que existen en el uso de estos sistemas, los cuales permiten
analizar posibles mejoras a realizar en futuras investigaciones.

La implementación del SRI mostrado dio resultados positivos, la sencillez del
modelo hace que su eficacia pruebe la fortaleza de este algoritmo. Se
evidencia la efectividad del modelo en la colecciones usadas, lo cual se
demuestra en las métricas usadas en la evaluación. Sin duda alguna en
el mundo cambiante de la RI se necesitaran de algoritmos más potentes y otras
mejoras en estos modelos para estar a la altura de la creciente era de la
información.
