\section{Introducción}\label{sec:intro}

El papel de la investigación en el desarrollo de la ciencia, más que evidente,
es su esencia misma. La \emph{Recuperación de Información} (RI)
no es una excepción, y su desarrollo progresivo como disciplina ha venido
marcada, para lo bueno y para lo malo, por los resultados de la investigación
realizada.\cite{chowdhury}

La RI es un campo donde se realiza una actividad importante de práctica
profesional y, también, de investigación científica. Por un lado, dio lugar a
la aparición de la industria de la información, con sus bases de datos y
sistemas de información, con un marcado componente práctico y profesional
centrado en el uso de las bases de datos para satisfacer las necesidades de
información de los usuarios. Por otro lado, la investigación científica ha
estado dirigida hacia el diseño de \emph{Sistemas de Recuperación de
Información} (SRI) más eficaces, dando lugar a diversas teorías, modelos y
experimentos en los que la evaluación ha ocupado un papel central. A lo largo
de la historia no ha existido una relación muy estrecha entre ambos
componentes, científico y profesional, al menos hasta los últimos años en los
que parece haberse producido un cierto acercamiento y una implementación de
las teorías y modelos experimentales en las aplicaciones comerciales.
\cite{ceri}

La investigación en RI ha favorecido sobre todo la estimulación de ideas, incluso
relacionadas con otros campos como la física cuántica \cite{piwowarski}. Estas
ideas, conforme se han ido explorando y comprobando en sistemas experimentales,
se han convertido en teorías o modelos. Es por ello que las teorías y
modelos desarrollados han estado influenciados en gran parte por la comprensión
y el conocimiento empírico obtenido a partir de experimentos en los que la
evaluación ha estado omnipresente.

Son muchos los experimentos e investigaciones llevadas a cabo en el campo de la
RI. También muchas son las críticas y discusiones sobre los resultados
obtenidos, derivadas, en su mayoría, por las diferentes metodologías
utilizadas. Esto se debe en parte a la gran variedad de problemas investigados
y gtambién a la diversa procedencia académica de los investigadores.
\cite{ellis}

En los SRI se han hecho grandes avances y se ha extendido su uso
de tal forma que nadie se imagina un día despertar y ver que grandes sistemas
de búsqueda en internet, como Google, no existan. Las necesidades crecientes
de la humanidad y la complejización que experimentan los problemas con el paso
del tiempo, fuerzan a la mejora de los modelos e implementaciones con el fin
de tener herramientas cada vez mejores. Se destacan en esta cuestión los
modelos básicos de los SRI: \emph{Modelo Booleano, Vectorial y Probabilístico},
siendo los principales, los cuales sentaron las bases de sistemas más robustos
desarrollados en la actualidad.


Uno de los subcampos más importantes dentro de la RI es la recuperación de
información textual \cite{boyce, blair}, la cual ha sido ampliamente abordada
desde los mismos inicios. El siguiente trabajo mostrará una implementación de un SRI
de texto utilizndo un modelo vectorial. Se mostrarán detalles de su diseño,
modelado, implementación, procesos por los que transita, evaluación así como
las ventajas y desventajas del mismo.
