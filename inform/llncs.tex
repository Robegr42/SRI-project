% This is LLNCS.DEM the demonstration file of
% the LaTeX macro package from Springer-Verlag
% for Lecture Notes in Computer Science,
% version 2.4 for LaTeX2e as of 16. April 2010
%
\documentclass{llncs}
%
\usepackage{makeidx}  % allows for indexgeneration
\usepackage{graphicx}
\usepackage[spanish]{babel}
\usepackage{url}
\usepackage[utf8]{inputenc}

\setcounter{tocdepth}{2}
%
\begin{document}
\frontmatter
\pagestyle{headings}  % switches on printing of running heads

\addtocmark[2]{Sistemas de Recuperación de Información\\ Trabajo Final}
\mainmatter              % start of the contributions
%
\title{Sistemas de Recuperación de Información\\ Trabajo Final}
%
\titlerunning{Sistemas de Recuperación de Información\\ Trabajo Final}  % abbreviated title (for running head)
%                                     also used for the TOC unless
%                                     \toctitle is used
%
\author{Jorge J. Morgado Vega\inst{1} \and Roberto García Rodrígez\inst{1}}
%
% \authorrunning{Ivar Ekeland et al.} % abbreviated author list (for running head)
%
%%%% list of authors for the TOC (use if author list has to be modified)
\tocauthor{Jorge J. Morgado Vega y Roberto García Rodrígez}
%
\institute{Universidad de la Habana, La Habana, Cuba}

\maketitle              % typeset the title of the contribution

\renewcommand\abstractname{Resúmen.}
\renewcommand\keywordname{{\bf Palabras Claves:}}
\begin{abstract}
	
\keywords{}
\end{abstract}
\renewcommand\abstractname{Abstract.}
\renewcommand\keywordname{{\bf Keywords:}}
\begin{abstract}
	
\keywords{}
\end{abstract}
%

\tableofcontents
\clearpage

\section{Introducción}\label{sec:intro}
\section{Introducción}\label{sec:intro}

El papel de la investigación en el desarrollo de la ciencia, más que evidente,
es su esencia misma. La \emph{Recuperación de Información} (RI)
no es una excepción, y su desarrollo progresivo como disciplina ha venido
marcada, para lo bueno y para lo malo, por los resultados de la investigación
realizada.\cite{chowdhury}

La RI es un campo donde se realiza una actividad importante de práctica
profesional y, también, de investigación científica. Por un lado, dio lugar a
la aparición de la industria de la información, con sus bases de datos y
sistemas de información, con un marcado componente práctico y profesional
centrado en el uso de las bases de datos para satisfacer las necesidades de
información de los usuarios. Por otro lado, la investigación científica ha
estado dirigida hacia el diseño de \emph{Sistemas de Recuperación de
Información} (SRI) más eficaces, dando lugar a diversas teorías, modelos y
experimentos en los que la evaluación ha ocupado un papel central. A lo largo
de la historia no ha existido una relación muy estrecha entre ambos
componentes, científico y profesional, al menos hasta los últimos años en los
que parece haberse producido un cierto acercamiento y una implementación de
las teorías y modelos experimentales en las aplicaciones comerciales.
\cite{ceri, borlund}

La investigación en RI ha favorecido sobre todo la estimulación de ideas, incluso
relacionadas con otros campos como la física cuántica \cite{piwowarski}. Estas
ideas, conforme se han ido explorando y comprobando en sistemas experimentales,
se han convertido en teorías o modelos. Es por ello que las teorías y
modelos desarrollados han estado influenciados en gran parte por la comprensión
y el conocimiento empírico obtenido a partir de experimentos en los que la
evaluación ha estado omnipresente.

Son muchos los experimentos e investigaciones llevadas a cabo en el campo de la
RI. También muchas son las críticas y discusiones sobre los resultados
obtenidos, derivadas, en su mayoría, por las diferentes metodologías
utilizadas. Esto se debe en parte a la gran variedad de problemas investigados
y gtambién a la diversa procedencia académica de los investigadores.
\cite{ellis}

En los SRI se han hecho grandes avances y se ha extendido su uso
de tal forma que nadie se imagina un día despertar y ver que grandes sistemas
de búsqueda en internet, como Google, no existan. Las necesidades crecientes
de la humanidad y la complejización que experimentan los problemas con el paso
del tiempo, fuerzan a la mejora de los modelos e implementaciones con el fin
de tener herramientas cada vez mejores. Se destacan en esta cuestión los
modelos básicos de los SRI: \emph{Modelo Booleano, Vectorial y Probabilístico},
siendo los principales, los cuales sentaron las bases de sistemas más robustos
desarrollados en la actualidad.


Uno de los subcampos más importantes dentro de la RI es la recuperación de
información textual \cite{boyce, blair}, la cual ha sido ampliamente abordada
desde los mismos inicios. El siguiente trabajo mostrará una implementación de un SRI
de texto utilizndo un modelo vectorial. Se mostrarán detalles de su diseño,
modelado, implementación, procesos por los que transita, evaluación así como
las ventajas y desventajas del mismo.


\section{Diseño e Implementación}\label{sec:design}
\section{Diseño e Implementación}\label{sec:design}

Para analizar el diseño de la aplicación se realizará una descripción de forma
\emph{top-down}. Primero se mostrarán las capas superiores comenzando por la
interacción del usuario con la aplicación y luego se analizará de forma
detallada el diseño e implementación de cada una de las componentes que
conforman la misma.

De forma general el usuario tiene dos formas de interactuar con la aplicación:
mediante una interfaz de lineas de comando en una terminal (\emph{Command Line
Interface}, CLI por sus siglas en inglés) desarrollada usando \emph{typer} o
mediante una interfaz gráfica desarrollada usando \emph{streamlit}.

Para hacer uso de la aplicación y comenzar a realizar consultas a una base de
datos, el usuario debe realizar primero dos acciones:

\begin{enumerate}
	\item Construir la base de datos.
	\item Construir el modelo de RI de la base de datos construida.
\end{enumerate}

\subsection{Construcción de la base de datos}\label{sec:build-database}

\subsection{Construcción del modelo}\label{sec:model}



\section{Evaluación}\label{sec:eval}
Como parte fundamental de la creación de un SRI se tiene el proceso de
evaluación, donde se determinará la eficacia del mismo.

Para la evaluación del modelo propuesto se cuenta con varias colecciones de
pruebas, cada una pertenecientes a las distintas colecciones de documentos con
las que el modelo trabaja entre las cuales el usuario puede escoger para
realizar sus consultas. Estás colecciones, a su vez, contienen un conjunto de
consultas y la información de los documentos relevantes y no relevantes, de
esta forma se determina el rendimiento del modelo.

Para el proceso de evaluación se analiza la siguiente estructura, resultante
del proceso de recuperación del modelo

\begin{figure}[htb]%
	\begin{center}
		\includegraphics[width=0.5\textwidth]{./sri_03.png}
	\end{center}
	\caption{Clasificación de los documentos después de una consulta.}
	\label{fig:docSet}
\end{figure}

Donde:
\begin{itemize}
    \item {\bf REL:} Conjunto de documentos relevantes.
    \item {\bf REC:} Conjunto de documentos recuperados.
    \item {\bf RR:} Conjunto de documentos relevantes recuperados.
    \item {\bf NN:} Conjunto de documentos no relevantes no recuperados.
\end{itemize}

Sobre esta partición de conjuntos se aplican varias medidas de evaluación, con
las cuales se verifican la efectividad del modelo, las mismas son:

\begin{itemize}
    \item Precisión
    \item Recobrado (Recall)
    \item Medida F
    \item Medida F1
\end{itemize}

Estas medidas, aunque efectivas, no tienen en cuanta el \emph{ranking} de los
documentos, por lo que para una mayor efectividad del modelo se emplean otras
dos medidas para un mejor rendimiento, las cuales son:

\begin{itemize}
    \item R-Precisión
    \item Fallout 
\end{itemize}

Se incluyen en el modelo, además, opciones de comparación entre varias
colecciones, con el fin de realizar análisis entre las mismas.

\section{Ventajas y Desventajas}\label{sec:pros_cons}
Como se ha podido apreciar en el desarrollo de este trabajo, \emph{tf-idf} es
un eficiente y simple algoritmo para hacer coincidir palabras en una consulta
con documentos que son relevantes para esa consulta \cite{ramos}. Como cada
modelo de RI, este tiene aspectos que lo hacen beneficioso, y al mismo tiempo,
a pesar de la fuerza del modelo, se escapan algunos puntos que le imponen
limitaciones.

\subsection{Ventajas}

\subsubsection{Esquema de Ponderación}

El modelo vectorial es muy versátil y eficiente a la hora de generar rankings
de precisión en colecciones de gran tamaño, lo que le hace idóneo para
determinar la equiparación parcial de los documentos. 

\subsubsection{Coincidencia Parcial}

La estrategia de coincidencia parcial permite la recuperación de documentos
que se aproximen a los requerimientos de la consulta. De esta forma quizas se
tengan documentos recuperados que no cumplan con todas las características de
la consulta, sino una parte considerable de la misma, los cuales son relevantes
a la consulta hecha.

\subsubsection{Medida de Similitud}

Se demuestra la eficacia de la \emph{fórmula del coseno} como métrica que
ordena los documentos de acuerdo al grado de similitud con la consulta,
aprovechádose de resultados \emph{geométrico-algebráicos} que plantean la
similitud entre vectores de acuerdo al coseno del ángulo que ellos determinan. 

\subsection{Desventajas}

\subsubsection{Estructura del Lenguaje}

Al ser un modelo estadístico-matemático, no tiene en cuenta la estructura
sintáctico-semántica del lenguaje natural. En general recupera documentos
cuando hay igualdad de palabras entre el documento y la consulta no tiene en
cuenta, por ejemplo, sinónimos.

\subsubsection{Umbral de Relevancia}

Establecer un umbral para definir que documentos son relevantes y cuales no es
un valor que no es fijo y fiable a cada colección, o sea, para cada colección
hay que probar y reajustar para ver que umbral se ajusta mejor y permite tener
métricas de evaluación más compensadas.



\section{Concluiones}\label{sec:conc}
\section{Conclusiones}\label{sec:conc}

En el trabajo presentado se mostraron las características de un SRI utilizando
una implementación de un \emph{Modelo Vectorial (tf-idf)}. Se explica su
diseño, siendo este, uno de las esquemas más usados en la actualidad. Se
detallan las distintas etapas del proceso desde la realización de una consulta,
y como esta es procesada, hasta la recuperación de los documentos relevantes a
la propia consulta. Se explican los algoritmos utilizados en la creación del
modelo así como el uso de varias colecciones de prueba para su evaluación.
Se debatieron, además, de forma ctítica, los principales beneficios y
limitantes que existen en el uso de estos sistemas, los cuales permiten
analizar posibles mejoras a realizar en futuras investigaciones.

La implementación del SRI mostrado dio resultados positivos, la sencillez del
modelo hace que su eficacia pruebe la fortaleza de este algoritmo. Se
evidencia la efectividad del modelo en la colecciones usadas, lo cual se
demuestra en las métricas usadas en la evaluación. Sin duda alguna en
el mundo cambiante de la RI se necesitaran de algoritmos más potentes y otras
mejoras en estos modelos para estar a la altura de la creciente era de la
información.


\section{Recomendaciones}\label{sec:rec}
Como recomendaciones al trabajo se pueden señalar varios aspectos interesantes
como: el uso e incorporación de retroalimentación al modelo con la cual se
ganaría en rendimiento y precisión del modelo; la integración de algoritmos de
\emph{Crawling} para una futura integración Web; operaciones textuales más
avanzadas y el uso de bases de conocimientos como ontologías.


%
% ---- Bibliography ----
%
\begin{thebibliography}{}
%

\end{thebibliography}
\clearpage
\addtocmark[2]{Subject Index} % additional numbered TOC entry
\markboth{Subject Index}{Subject Index}
\renewcommand{\indexname}{Subject Index}
%                                                           clmomu01.ind
%-----------------------------------------------------------------------
% CLMoMu01 1.0: LaTeX style files for books
% Sample index file for User's guide
% (c) Springer-Verlag HD
%-----------------------------------------------------------------------
\begin{theindex}

\item Aplicaciones implementadas \idxquad 7-8
\item API \idxquad 7
\item Construccion de un modelo \idxquad 5-6
\item Desventajas \idxquad 11-12
\item Estructura de las colecciones \idxquad 8
\item Procesamiento de una consulta \idxquad6
\item Métricas de evaluación \idxquad 8-9
\subitem Precisión \idxquad 8 
\subitem Recobrado \idxquad 9
\subitem Medida $F_1$ \idxquad 9
\subitem Fallout \idxquad 9
\item TF-IDF \idxquad 4
\item Ventajas \idxquad 10-11

\end{theindex}

\end{document}
