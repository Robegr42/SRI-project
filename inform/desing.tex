\section{Diseño e Implementación}\label{sec:design}

Para analizar el diseño de la aplicación se realizará una descripción de forma
\emph{top-down}. Primero se mostrarán las capas superiores comenzando por la
interacción del usuario con la aplicación y luego se analizará de forma
detallada el diseño e implementación de cada una de las componentes que
conforman la misma.

De forma general el usuario tiene dos formas de interactuar con la aplicación:
mediante una interfaz de lineas de comando en una terminal (\emph{Command Line
Interface}, CLI por sus siglas en inglés) desarrollada usando \emph{typer} o
mediante una interfaz gráfica desarrollada usando \emph{streamlit}.

Para hacer uso de la aplicación y comenzar a realizar consultas a una base de
datos, el usuario debe realizar primero dos acciones:

\begin{enumerate}
	\item Construir la base de datos.
	\item Construir el modelo de RI de la base de datos construida.
\end{enumerate}

\subsection{Construcción de la base de datos}\label{sec:build-database}

\subsection{Construcción del modelo}\label{sec:model}

